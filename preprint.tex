%!TEX TS-program = xelatex
\documentclass[10pt,oneside]{article}
\usepackage[fontsize=9pt]{scrextend}

\usepackage[english]{babel}

\usepackage{amsmath,amssymb,amsfonts}
\usepackage[utf8]{inputenc}
\usepackage[T1]{fontenc}
\usepackage{stix2}
\usepackage[scaled]{helvet}
\usepackage[scaled]{inconsolata}

\usepackage{lastpage}

\usepackage{setspace}

\usepackage{ccicons}

\usepackage[hang,flushmargin]{footmisc}

\usepackage{geometry}

\setlength{\parindent}{0pt}
\setlength{\parskip}{6pt plus 2pt minus 1pt}

\usepackage{fancyhdr}
\renewcommand{\headrulewidth}{0pt}\providecommand{\tightlist}{%
  \setlength{\itemsep}{0pt}\setlength{\parskip}{0pt}}

\makeatletter
\newcounter{tableno}
\newenvironment{tablenos:no-prefix-table-caption}{
  \caption@ifcompatibility{}{
    \let\oldthetable\thetable
    \let\oldtheHtable\theHtable
    \renewcommand{\thetable}{tableno:\thetableno}
    \renewcommand{\theHtable}{tableno:\thetableno}
    \stepcounter{tableno}
    \captionsetup{labelformat=empty}
  }
}{
  \caption@ifcompatibility{}{
    \captionsetup{labelformat=default}
    \let\thetable\oldthetable
    \let\theHtable\oldtheHtable
    \addtocounter{table}{-1}
  }
}
\makeatother

\usepackage{array}
\newcommand{\PreserveBackslash}[1]{\let\temp=\\#1\let\\=\temp}
\let\PBS=\PreserveBackslash

\usepackage[breaklinks=true]{hyperref}
\hypersetup{colorlinks,%
citecolor=blue,%
filecolor=blue,%
linkcolor=blue,%
urlcolor=blue}
\usepackage{url}

\usepackage{caption}
\setcounter{secnumdepth}{0}
\usepackage{cleveref}

\usepackage{graphicx}
\makeatletter
\def\maxwidth{\ifdim\Gin@nat@width>\linewidth\linewidth
\else\Gin@nat@width\fi}
\makeatother
\let\Oldincludegraphics\includegraphics
\renewcommand{\includegraphics}[1]{\Oldincludegraphics[width=\maxwidth]{#1}}

\usepackage{longtable}
\usepackage{booktabs}

\usepackage{color}
\usepackage{fancyvrb}
\newcommand{\VerbBar}{|}
\newcommand{\VERB}{\Verb[commandchars=\\\{\}]}
\DefineVerbatimEnvironment{Highlighting}{Verbatim}{commandchars=\\\{\}}
% Add ',fontsize=\small' for more characters per line
\usepackage{framed}
\definecolor{shadecolor}{RGB}{248,248,248}
\newenvironment{Shaded}{\begin{snugshade}}{\end{snugshade}}
\newcommand{\KeywordTok}[1]{\textcolor[rgb]{0.13,0.29,0.53}{\textbf{#1}}}
\newcommand{\DataTypeTok}[1]{\textcolor[rgb]{0.13,0.29,0.53}{#1}}
\newcommand{\DecValTok}[1]{\textcolor[rgb]{0.00,0.00,0.81}{#1}}
\newcommand{\BaseNTok}[1]{\textcolor[rgb]{0.00,0.00,0.81}{#1}}
\newcommand{\FloatTok}[1]{\textcolor[rgb]{0.00,0.00,0.81}{#1}}
\newcommand{\ConstantTok}[1]{\textcolor[rgb]{0.00,0.00,0.00}{#1}}
\newcommand{\CharTok}[1]{\textcolor[rgb]{0.31,0.60,0.02}{#1}}
\newcommand{\SpecialCharTok}[1]{\textcolor[rgb]{0.00,0.00,0.00}{#1}}
\newcommand{\StringTok}[1]{\textcolor[rgb]{0.31,0.60,0.02}{#1}}
\newcommand{\VerbatimStringTok}[1]{\textcolor[rgb]{0.31,0.60,0.02}{#1}}
\newcommand{\SpecialStringTok}[1]{\textcolor[rgb]{0.31,0.60,0.02}{#1}}
\newcommand{\ImportTok}[1]{#1}
\newcommand{\CommentTok}[1]{\textcolor[rgb]{0.56,0.35,0.01}{\textit{#1}}}
\newcommand{\DocumentationTok}[1]{\textcolor[rgb]{0.56,0.35,0.01}{\textbf{\textit{#1}}}}
\newcommand{\AnnotationTok}[1]{\textcolor[rgb]{0.56,0.35,0.01}{\textbf{\textit{#1}}}}
\newcommand{\CommentVarTok}[1]{\textcolor[rgb]{0.56,0.35,0.01}{\textbf{\textit{#1}}}}
\newcommand{\OtherTok}[1]{\textcolor[rgb]{0.56,0.35,0.01}{#1}}
\newcommand{\FunctionTok}[1]{\textcolor[rgb]{0.00,0.00,0.00}{#1}}
\newcommand{\VariableTok}[1]{\textcolor[rgb]{0.00,0.00,0.00}{#1}}
\newcommand{\ControlFlowTok}[1]{\textcolor[rgb]{0.13,0.29,0.53}{\textbf{#1}}}
\newcommand{\OperatorTok}[1]{\textcolor[rgb]{0.81,0.36,0.00}{\textbf{#1}}}
\newcommand{\BuiltInTok}[1]{#1}
\newcommand{\ExtensionTok}[1]{#1}
\newcommand{\PreprocessorTok}[1]{\textcolor[rgb]{0.56,0.35,0.01}{\textit{#1}}}
\newcommand{\AttributeTok}[1]{\textcolor[rgb]{0.77,0.63,0.00}{#1}}
\newcommand{\RegionMarkerTok}[1]{#1}
\newcommand{\InformationTok}[1]{\textcolor[rgb]{0.56,0.35,0.01}{\textbf{\textit{#1}}}}
\newcommand{\WarningTok}[1]{\textcolor[rgb]{0.56,0.35,0.01}{\textbf{\textit{#1}}}}
\newcommand{\AlertTok}[1]{\textcolor[rgb]{0.94,0.16,0.16}{#1}}
\newcommand{\ErrorTok}[1]{\textcolor[rgb]{0.64,0.00,0.00}{\textbf{#1}}}
\newcommand{\NormalTok}[1]{#1}

\newlength{\cslhangindent}
\setlength{\cslhangindent}{1.5em}
\newlength{\csllabelwidth}
\setlength{\csllabelwidth}{3em}
\newenvironment{CSLReferences}[3] % #1 hanging-ident, #2 entry spacing
 {% don't indent paragraphs
  \setlength{\parindent}{0pt}
  % turn on hanging indent if param 1 is 1
  \ifodd #1 \everypar{\setlength{\hangindent}{\cslhangindent}}\ignorespaces\fi
  % set entry spacing
  \ifnum #2 > 0
  \setlength{\parskip}{#2\baselineskip}
  \fi
 }%
 {}
\usepackage{calc} % for \widthof, \maxof
\newcommand{\CSLBlock}[1]{#1\hfill\break}
\newcommand{\CSLLeftMargin}[1]{\parbox[t]{\maxof{\widthof{#1}}{\csllabelwidth}}{#1}}
\newcommand{\CSLRightInline}[1]{\parbox[t]{\linewidth}{#1}}
\newcommand{\CSLIndent}[1]{\hspace{\cslhangindent}#1}\usepackage[table,dvipsnames]{xcolor}

\geometry{includemp,
            letterpaper,
            top=2.4cm,
            bottom=2.4cm,
            left=1.0cm,
            right=1.0cm,
            marginparwidth=5cm,
            marginparsep=1.0cm}

\usepackage[singlelinecheck=off]{caption}

\captionsetup{
  font={small},
  labelfont={bf},
  format=plain,
  labelsep=quad
}

\usepackage{floatrow}

\floatsetup[figure]{margins=hangright,
              facing=no,
              capposition=beside,
              capbesideposition={center,outside},
              floatwidth=\textwidth}

% \floatsetup[table]{margins=hangright,
%              facing=no,
%              capposition=beside,
%              capbesideposition={center,outside},
%              floatwidth=\textwidth}

\pagestyle{plain}

\setcounter{secnumdepth}{5}

\usepackage{titlesec}

\titleformat{\section}[block]
{\normalfont\large\sffamily}
{\thesection}{.5em}{\titlerule\\[.8ex]\bfseries}

\titleformat{\subsection}[runin]
{\normalfont\fontseries{b}\selectfont\filright\sffamily}
{\thesubsection.}{.5em}{}

\titleformat{\subsubsection}[runin]
{\normalfont\itshape\rmfamily\bfseries}{\thesubsubsection}{1em}{}

\fancypagestyle{firstpage}
{
   \fancyhf{}
   \renewcommand{\headrulewidth}{0pt}
   \fancyfoot[R]{\footnotesize\ccby}
   \fancyfoot[L]{\footnotesize\sffamily\today}
}

\fancypagestyle{normal}
{
  \fancyhf{}
  \fancyfoot[R]{\footnotesize\sffamily\thepage\ of \pageref*{LastPage}}
}

\usepackage{tikz}
\begin{document}
\tikz [remember picture, overlay] %
\node [shift={(-0.6in,1.1cm)},scale=0.2,opacity=0.4] at (current page.south east)[anchor=south east]{\includegraphics{logo}};%
\pagestyle{normal}
\thispagestyle{firstpage}

\newcommand{\colorRule}[3][black]{\textcolor[HTML]{#1}{\rule{#2}{#3}}}

\noindent {\LARGE \textbf{\textsf{Inter-ecosystem specifications of
energy transfer in trophic interactions}}}

\medskip
\begin{flushleft}
{\small
%
\href{https://orcid.org/0000-0002-4104-9463}{Benjamin\,Mercier}%
%
\,\textsuperscript{1,2,‡}
\vskip 1em
\textsuperscript{1}\,Université de
Sherbrooke; \textsuperscript{2}\,Québec Centre for Biodiversity
Sciences\\
\textsuperscript{‡}\,These authors contributed equally to the work\\
\vskip 1em
\textbf{Correspondance to:}\\
Benjamin Mercier --- \texttt{benjamin.b.mercier@usherbrooke.ca}\\
}
\end{flushleft}

\vskip 2em
\makebox[0pt][l]{\colorRule[CCCCCC]{2.0\textwidth}{0.5pt}}
\vskip 2em
\noindent

\marginpar{\vskip 1em\flushright
{\small{\bfseries Keywords}:\par
trophic interactions\\energy fluxes\\ecosystem\\}
}


This is a very short abstract




\vskip 2em
\makebox[0pt][l]{\colorRule[CCCCCC]{2.0\textwidth}{0.5pt}}
\vskip 2em

\hypertarget{introduction}{%
\section{Introduction}\label{introduction}}

Blabla this is an introduction. This is a citation test Brose \emph{et
al.} (2019), and this is also another citation test (Brose \emph{et al.}
2019).

\hypertarget{the-data}{%
\section{The data}\label{the-data}}

The Ecopath data were obtain from Jacquet \emph{et al.} (2016) on
request to the corresponding author. Before manipulation, the data
initially represented 116 Ecopath trophic networks. Ecopath is a
modeling software which aims to quantify species interactions statically
and is mass-balanced (Christensen n.d.). One weakness, if I may, is that
a lot of these Ecopath networks are not taxonomically resolved to the
species but encompass trophic groups or guilds. The first step here was
then, for each of these networks, match the data to the original article
from which they originated. This matching was done to

\hypertarget{figures}{%
\section{Figures}\label{figures}}

\begin{figure}
\hypertarget{fig:figure}{%
\centering
\includegraphics{figures/flux_ppmr.png}
\caption{This is a prelim figure about the fluxes.}\label{fig:figure}
}
\end{figure}

\begin{figure}
\hypertarget{fig:map}{%
\centering
\includegraphics{figures/network_map.png}
\caption{This is the map of network}\label{fig:map}
}
\end{figure}

\hypertarget{analyses}{%
\section{Analyses}\label{analyses}}

\hypertarget{conclusion}{%
\section*{Conclusion}\label{conclusion}}
\addcontentsline{toc}{section}{Conclusion}

\hypertarget{refs}{}
\begin{CSLReferences}{1}{0}
\leavevmode\hypertarget{ref-Brose2019PreTra}{}%
Brose, U., Archambault, P., Barnes, A.D., Bersier, L.-F., Boy, T.,
Canning-Clode, J., \emph{et al.} (2019). Predator traits determine
food-web architecture across ecosystems. \emph{Nature Ecology \&
Evolution}, 3, 919--927.

\leavevmode\hypertarget{ref-ChristensenEcoEco}{}%
Christensen, V. (n.d.). Ecopath with Ecosim: A User's Guide, 155.

\leavevmode\hypertarget{ref-Jacquet2016NoCom}{}%
Jacquet, C., Moritz, C., Morissette, L., Legagneux, P., Massol, F.,
Archambault, P., \emph{et al.} (2016). No complexitystability
relationship in empirical ecosystems. \emph{Nature Communications}, 7,
12573.

\end{CSLReferences}

\end{document}
